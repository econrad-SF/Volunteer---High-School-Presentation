%%%%%%%%%%%%%%%%%%%%%%%%%%%%%%%%%%%%%%%%%
% Beamer Presentation
% LaTeX Template
% Version 1.0 (10/11/12)
%
% This template has been downloaded from:
% http://www.LaTeXTemplates.com
%
% License:
% CC BY-NC-SA 3.0 (http://creativecommons.org/licenses/by-nc-sa/3.0/)
%
%%%%%%%%%%%%%%%%%%%%%%%%%%%%%%%%%%%%%%%%%

%----------------------------------------------------------------------------------------
%	PACKAGES AND THEMES
%----------------------------------------------------------------------------------------

\documentclass[t]{beamer} % specify vertical top alignment globally with 't'

\mode<presentation> {

% The Beamer class comes with a number of default slide themes
% which change the colors and layouts of slides. Below this is a list
% of all the themes, uncomment each in turn to see what they look like.

%\usetheme{default}
%\usetheme{AnnArbor}
%\usetheme{Antibes}
%\usetheme{Bergen}
%\usetheme{Berkeley}
%\usetheme{Berlin}
%\usetheme{Boadilla}   %%%%%%%
%\usetheme{CambridgeUS}
%\usetheme{Copenhagen}
%\usetheme{Darmstadt}
%\usetheme{Dresden}
%\usetheme{Frankfurt}
%\usetheme{Goettingen}
%\usetheme{Hannover}
%\usetheme{Ilmenau}
%\usetheme{JuanLesPins}
%\usetheme{Luebeck}
\usetheme{Madrid}  %%%%%
%\usetheme{Malmoe}
%\usetheme{Marburg}
%\usetheme{Montpellier}
%usetheme{PaloAlto}
%\usetheme{Pittsburgh}
%\usetheme{Rochester}
%\usetheme{Singapore}
%\usetheme{Szeged}
%\usetheme{Warsaw}

% As well as themes, the Beamer class has a number of color themes
% for any slide theme. Uncomment each of these in turn to see how it
% changes the colors of your current slide theme.

%\usecolortheme{albatross}
%\usecolortheme{beaver}
%\usecolortheme{beetle}
%\usecolortheme{crane}
%\usecolortheme{dolphin}
%\usecolortheme{dove}
%\usecolortheme{fly}
%\usecolortheme{lily}
%\usecolortheme{orchid}
%\usecolortheme{rose}
%\usecolortheme{seagull}
%\usecolortheme{seahorse}
%\usecolortheme{whale}
%\usecolortheme{wolverine}

%\setbeamertemplate{footline} % To remove the footer line in all slides uncomment this line
%\setbeamertemplate{footline}[page number] % To replace the footer line in all slides with a simple slide count uncomment this line

\setbeamertemplate{navigation symbols}{} % To remove the navigation symbols from the bottom of all slides uncomment this line
}
\usepackage{animate}    % migration graphics that loop through 12 months
\usepackage{media9}    % video
\usepackage{gensymb}  % allows \degree
\usepackage{graphicx} % Allows including images
\usepackage{booktabs} % Allows the use of \toprule, \midrule and \bottomrule in tables
\graphicspath { {images/} } % specify the location of where images are kept under current directory
\usepackage[export]{adjustbox}
\usepackage[font=small, labelfont=bf, skip=-8pt]{caption} % Required for specifying captions to tables and figures
\usepackage{dirtytalk}  % For better looking quotation marks.
\usepackage{hyperref}

 
\definecolor{Hummingbird_red}{RGB}{255, 25, 25} % Create new green color using Red/Green/Blue values.
\usecolortheme[named=Hummingbird_red]{structure} % Change default blue colored tiles from "Madrid" to red


\setbeamerfont{myTOC}{series=\bfseries,size=\large}
\AtBeginSection[]{\frame{\frametitle{Outline}%
\usebeamerfont{myTOC}\tableofcontents[current]}}

%----------------------------------------------------------------------------------------
%	TITLE PAGE
%----------------------------------------------------------------------------------------

\title[Hummingbirds of New Mexico]{Hummingbirds of New Mexico} % The short title appears at the bottom of every slide, the full title is only on the title page

\author{Ed Conrad} % Your name
\institute[] % Your institution as it will appear on the bottom of every slide, may be shorthand to save space
{
%New Mexico State Forestry Division \\ % Your institution for the title page
\medskip
%\textit{edward.conrad@state.nm.us} % Your email address
}
\date{\today} % Date, can be changed to a custom date

\begin{document}

\begin{frame}
\titlepage % Print the title page as the first slide
\end{frame}

%\begin{frame}
%\frametitle{Overview} % Table of contents slide, comment this block out to remove it
%\tableofcontents % Throughout your presentation, if you choose to use \section{} and \subsection{} commands, these will automatically be printed on this slide as an overview of your presentation
%\end{frame}

%----------------------------------------------------------------------------------------
%	PRESENTATION SLIDES
%----------------------------------------------------------------------------------------
%\section{} % Sections can be created in order to organize your presentation into discrete blocks, all sections and subsections are automatically printed in the table of contents as an overview of the talk
%\subsection{} % A subsection can be created just before a set of slides with a common theme to further break down your presentation into chunks
%\begin{frame}
%\frametitle{Class Aves}
%\begin{itemize}
%	\item 10,500 species of birds on Earth\\
%	
%Kingdom: Animalia
%Phylum: Chordata
%Class: Aves
%Order: Apod
%	
%	
%	\item Trochilidae \say{small bird} (2nd largest Family)
%\end{itemize}
%\end{frame}

%------------------------------------------------

\begin{frame}
\frametitle{Hummingbirds of the World -- \small{320 species in Family Trochilidae}}
\vspace{-0.3in}
\begin{table}
\begin{tabular}{l l l}
\textbf{1.}\includegraphics[scale=0.45, valign=T]{world/Sword_billed} & \textbf{2.}\includegraphics[scale=0.23, valign=T]{world/Rufous_crested_Coquette} \textbf{3.}\includegraphics[scale=0.15, valign=T]{world/Velvet_breasted_Coronet} \\ 
\end{tabular}
\end{table}
\textbf{4.}\includegraphics[scale=0.24, valign=T]{world/Red-billed-Streamertail} \\
\end{frame}

%------------------------------------------------

\begin{frame}
\frametitle{Hummingbirds of the World -- \small{320 species in Family Trochilidae}}
\vspace{-0.3in}
\begin{table}
\begin{tabular}{l l l}
\textbf{1.}\includegraphics[scale=0.065, valign=T]{world/booted_racket_tail} & \textbf{2.}\includegraphics[scale=0.15, valign=T]{world/sapphire_vented_puffleg}
\end{tabular}
\end{table}
\textbf{3.}\includegraphics[scale=0.70, valign=T]{world/crimson_topaz}\textbf{4.}\includegraphics[scale=0.20, valign=T]{world/violet_tailed_sylph}\textbf{5.}\includegraphics[scale=0.20, valign=T]{world/green_breasted_mango}\textbf{6.}\includegraphics[scale=1.0, valign=T]{world/Fiery_throated}

\end{frame}

%------------------------------------------------

\begin{frame}
\frametitle{Hummingbird Metabolism \& Torpor}
\vspace{-0.3in}
\begin{center}
\href{https://nyti.ms/2GKRkZh}{\includegraphics[scale=1.3, valign=T]{talk/b8HummersOnFeeder}}
\end{center}
\end{frame}

%------------------------------------------------

\begin{frame}
\frametitle{Hummingbirds' Feeding Habits \& Tongues}
\vspace{-0.3in}
\begin{center}
\includegraphics[scale=0.30, valign=T]{talk/hummingbird_tongue2}\\
\href{https://nyti.ms/2k4Pw1j}{\includegraphics[scale=0.4, valign=T]{talk/hummingbird_tongue3}}

\end{center}
\end{frame}

%------------------------------------------------

\begin{frame}
\frametitle{Hummingbirds' Breeding Season}
\vspace{-0.3in}
\begin{center}
\includegraphics[scale=0.75, valign=T]{talk/annashumming_dive}
\includegraphics[scale=0.25, valign=T]{talk/RUHU_nest}
\end{center}
\end{frame}

%------------------------------------------------

\begin{frame}
\frametitle{Hummingbird Research - Mist Netting \& Banding}
\vspace{-0.3in}
\begin{table}
\begin{tabular}{l l l}
\textbf{1.}\includegraphics[scale=0.25, valign=T]{talk/mistnet} & \textbf{2.}\includegraphics[scale=0.15, valign=T]{talk/hummingbird_mistnet} 
\end{tabular}
\end{table}
\textbf{3.}\includegraphics[scale=0.40, valign=T]{talk/RUHU_band}
\end{frame}

%------------------------------------------------

\begin{frame}
\frametitle{Hummingbird Research - Mist Netting \& Banding}

\begin{table}
\begin{tabular}{l l l}
\textbf{1.}\includegraphics[scale=0.15, valign=T]{talk/fiery} & \textbf{2.}\includegraphics[scale=0.15, valign=T]{talk/magnificent} \textbf{3.}\includegraphics[scale=0.15, valign=T]{talk/violet_ear} \\ 
\textbf{4.}\includegraphics[scale=0.11, valign=T]{talk/btah_male} & \textbf{5.}\includegraphics[scale=0.15, valign=T]{talk/mountain_gem} \textbf{6.}\includegraphics[scale=0.15, valign=T]{talk/purple_crowned_fairy}
\end{tabular}
\end{table}
\end{frame}

%------------------------------------------------

\begin{frame}
\frametitle{Hummingbird Field Identification can be Tough!}
\vspace{-0.3in}
\begin{center}
\includegraphics[scale=0.65, valign=T]{id/unhu}
\end{center}
\end{frame}

%------------------------------------------------

\begin{frame}
\frametitle{Black-chinned Hummingbird \small\textit{Archilochus alexandri}}
\vspace{-0.15in}
\begin{table}
\begin{tabular}{l l}
\textbf{1.}\includegraphics[scale=0.13, valign=T]{id/BCHU_male2} & \textbf{2.}\includegraphics[scale=0.2, valign=T]{id/BCHU_male1} \\
\textbf{3.}\includegraphics[scale=0.2, valign=T]{id/BCHU_male0} & \textbf{4.}\includegraphics[scale=0.15, valign=T]{id/BCHU_male3} \\
\end{tabular}
\end{table}
\end{frame}

%------------------------------------------------

\begin{frame}
\frametitle{Black-chinned Hummingbird \small\textit{Archilochus alexandri}}
\vspace{-0.15in}
\begin{table}
\begin{tabular}{l l}
\textbf{1.}\includegraphics[scale=0.2, valign=T]{id/BCHU_f_type0} & \textbf{2.}\includegraphics[scale=0.2, valign=T]{id/BCHU_f_type1} \\
\textbf{3.}\includegraphics[scale=0.2, valign=T]{id/BCHU_f_type2} & \textbf{4.}\includegraphics[scale=0.12, valign=T]{id/BCHU_f_type3} \\
\end{tabular}
\end{table}
\end{frame}

%------------------------------------------------

\begin{frame}
\frametitle{Black-chinned Hummingbird \small\textit{Archilochus alexandri}}
\hspace{0.5in}\includegraphics[scale=0.22, valign=T]{BCHU_range_map}
\end{frame}

%------------------------------------------------
\begin{frame}
\begin{center}
\animategraphics[autoplay,loop,scale=0.4]{0.85}{ebird/BCHU}{1}{12}
\end{center}
\end{frame}

%------------------------------------------------

\begin{frame}
\frametitle{Broad-tailed Hummingbird \small\textit{Selasphorous platycercus}}
\vspace{-0.25in}
\begin{table}
\begin{tabular}{l l}
\textbf{1.}\includegraphics[scale=0.2, valign=T]{id/BTAH_male2} & \textbf{2.}\includegraphics[scale=0.2, valign=T]{id/BTAH_male1} \\
\textbf{3.}\includegraphics[scale=0.08, valign=T]{id/BTAH_male3} & \textbf{4.}\includegraphics[scale=0.2, valign=T]{id/BTAH_male0} \\
\end{tabular}
\end{table}
\end{frame}

%------------------------------------------------

\begin{frame}
\frametitle{Broad-tailed Hummingbird \small\textit{Selasphorous platycercus}}
\vspace{-0.15in}
\begin{table}
\begin{tabular}{l l}
\textbf{1.}\includegraphics[scale=0.16, valign=T]{id/BTAH_f_type3} & \textbf{2.}\includegraphics[scale=0.2, valign=T]{id/BTAH_f_type1} \\
\textbf{3.}\includegraphics[scale=.2, valign=T]{id/BTAH_f_type0} & \textbf{4.}\includegraphics[scale=2, valign=T]{id/BTAH_f_type2} \\
\end{tabular}
\end{table}
\end{frame}

%------------------------------------------------

\begin{frame}
\frametitle{Broad-tailed Hummingbird \small\textit{Selasphorous platycercus}}
\hspace{1.0in}\includegraphics[scale=0.22, valign=T]{BTAH_range_map}
\end{frame}

%------------------------------------------------
\begin{frame}
\begin{center}
\animategraphics[autoplay,loop,scale=0.40]{0.85}{ebird/BTAH}{1}{12}
\end{center}
\end{frame}

%------------------------------------------------
\begin{frame}
\frametitle{Rufous Hummingbird \small\textit{Selasphorous rufus}}
\vspace{-0.15in}
\begin{table}
\begin{tabular}{l l}
\textbf{1.}\includegraphics[scale=0.2, valign=T]{id/RUHU_male0} & \textbf{2.}\includegraphics[scale=0.2, valign=T]{id/RUHU_male1} \\
\textbf{3.}\includegraphics[scale=0.2, valign=T]{id/RUHU_male3} & \textbf{4.}\includegraphics[scale=0.2, valign=T]{id/RUHU_male2} \\
\end{tabular}
\end{table}
\end{frame}

%------------------------------------------------

\begin{frame}
\frametitle{Rufous Hummingbird \small\textit{Selasphorous rufus}}
\vspace{-0.15in}
\begin{table}
\begin{tabular}{l l}
\textbf{1.}\includegraphics[scale=0.13, valign=T]{id/RUHU_f_type3} & \textbf{2.}\includegraphics[scale=0.2, valign=T]{id/RUHU_f_type1} \\
\textbf{3.}\includegraphics[scale=0.2, valign=T]{id/RUHU_f_type0} & \textbf{4.}\includegraphics[scale=0.13, valign=T]{id/RUHU_f_type2} \\
\end{tabular}
\end{table}
\end{frame}

%------------------------------------------------

\begin{frame}
\frametitle{Rufous Hummingbird \small\textit{Selasphorous rufus}}
\hspace{1.0in}\includegraphics[scale=0.135, valign=T]{RUHU_range_map}
\end{frame}

%------------------------------------------------
\begin{frame}
\begin{center}
\animategraphics[autoplay,loop,scale=0.40]{0.85}{ebird/RUHU}{1}{12}
\end{center}
\end{frame}

%------------------------------------------------

\begin{frame}
\frametitle{Calliope Hummingbird \small\textit{Selasphorous calliope}}
\vspace{-0.15in}
\begin{table}
\begin{tabular}{l l}
\textbf{1.}\includegraphics[scale=0.2, valign=T]{id/CAHU_male0} & \textbf{2.}\includegraphics[scale=0.11, valign=T]{id/CAHU_male1} \\
\textbf{3.}\includegraphics[scale=0.15, valign=T]{id/CAHU_male3} & \textbf{4.}\includegraphics[scale=0.2, valign=T]{id/CAHU_male2} \\
\end{tabular}
\end{table}
\end{frame}

%------------------------------------------------

\begin{frame}
\frametitle{Calliope Hummingbird \small\textit{Selasphorous calliope}}
\vspace{-0.15in}
\begin{table}
\begin{tabular}{l l}
\textbf{1.}\includegraphics[scale=0.2, valign=T]{id/CAHU_f_type0} & \textbf{2.}\includegraphics[scale=0.11, valign=T]{id/CAHU_f_type1} \\
\textbf{3.}\includegraphics[scale=0.2, valign=T]{id/CAHU_f_type2} & \textbf{4.}\includegraphics[scale=0.15, valign=T]{id/CAHU_f_type3} \\
\end{tabular}
\end{table}
\end{frame}

%------------------------------------------------

\begin{frame}
\frametitle{Calliope Hummingbird \small\textit{Selasphorous calliope}}
\hspace{1.0in}\includegraphics[scale=0.24, valign=T]{CAHU_range_map}
\end{frame}

%------------------------------------------------
\begin{frame}
\begin{center}
\animategraphics[autoplay,loop,scale=0.40]{0.85}{ebird/CAHU}{1}{12}
\end{center}
\end{frame}

%------------------------------------------------

\begin{frame}
\frametitle{New Mexico Hummingbirds}
\footnotesize\textbf{1. Black-chinned Hummingbird}\hspace{0.7in}\textbf{2. Broad-tailed Hummingbird}
\vspace{-0.1in}
\begin{table}
\begin{tabular}{l l}
\animategraphics[autoplay,loop,scale=0.32]{0.85}{ebird_NM/BCHU_NM}{1}{12} & \animategraphics[autoplay,loop,scale=0.32]{0.85}{ebird_NM/BTAH_NM}{1}{12} \\
\end{tabular}
\end{table}
\vspace{-0.15in}
\end{frame}

%------------------------------------------------

\begin{frame}
\frametitle{New Mexico Hummingbirds}
\footnotesize\textbf{3. Rufous Hummingbird}\hspace{1.1in}\textbf{4. Calliope Hummingbird}
\vspace{-0.1in}
\begin{table}
\begin{tabular}{l l}
\animategraphics[autoplay,loop,scale=0.32]{0.85}{ebird_NM/RUHU_NM}{1}{12} & \animategraphics[autoplay,loop,scale=0.32]{0.85}{ebird_NM/CAHU_NM}{1}{12} \\
\end{tabular}
\end{table}
\vspace{-0.15in}
\end{frame}

%----------------------------------------------------------------------------------------

\begin{frame}
\frametitle{Any Questions?}
\vspace{-0.2in}
\begin{center}
\includegraphics[scale=0.15, valign=T]{talk/bthu_shorts}
\end{center}
\end{frame}
%----------------------------------------------------------------------------------------
\end{document} 